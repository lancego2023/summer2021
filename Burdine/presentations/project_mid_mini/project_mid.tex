\documentclass[10pt]{beamer}

%% Based on the original theme by Matthias Vogelgesang
%% 		https://github.com/hsrmbeamertheme/hsrmbeamertheme
%% Also Based on the UNC Charlotte theme by Benjamin Radford
%% 		https://github.com/matze/mtheme
%% This theme is licensed under a Creative Commons Atribution-ShareAlike 4.0 international license
%% For more info, see http://creativecommons.org/licenses/by-sa/4.0/
%%
%% Source obtained from:
%% https://www.overleaf.com/latex/templates/unc-charlotte-beamer-theme/rnwqwjmrnmsk

\usetheme[progressbar=frametitle]{metropolis}
\usepackage{appendixnumberbeamer}
\usepackage{booktabs}
\usepackage[scale=2]{ccicons}
\usepackage{pgfplots}
\usepgfplotslibrary{dateplot}

\usepackage{xspace}
\newcommand{\themename}{\textbf{\textsc{metropolis}}\xspace}
\newcommand{\tabimg}[2]{\begin{tabular}{c}\includegraphics[scale=#2]{#1}\end{tabular}}


%%%%%%%%%%%%%%%%%%%%%%%%%%%%
%% Theme Adjustments %%
%%%%%%%%%%%%%%%%%%%%%%%%%%%%
\definecolor{CanvasBG}{HTML}{FAFAFA}

% Supporting Color Palette
\definecolor{White}{HTML}{FFFFFF}
\definecolor{WarmGray}{HTML}{696158}
\definecolor{StoneGray}{HTML}{717C7D}
\definecolor{DarkGreen}{HTML}{13381a}
\definecolor{LightGreen}{HTML}{509E2F}
\definecolor{BrightGold}{HTML}{F0CB00}
\definecolor{Royal}{HTML}{72246C}
\definecolor{Ocean}{HTML}{006BA6}
\definecolor{Flash}{HTML}{B52555}
\definecolor{Ruby}{HTML}{b50e0e}
\definecolor{Citrus}{HTML}{FFB81C}
\definecolor{Spring}{HTML}{CEDC00}
\definecolor{Garden}{HTML}{B7CE95}
\definecolor{Sand}{HTML}{F0E991}
\definecolor{Bloom}{HTML}{F1E6B2}
\definecolor{Clay}{HTML}{B7B09C}
\definecolor{Steel}{HTML}{b7b5ba}
\definecolor{Meteorite}{HTML}{404040}
\definecolor{Cloud}{HTML}{BAC5B9}


% Set colors here
\setbeamercolor{frametitle}{bg=Meteorite}
\setbeamercolor{progress bar}{bg=Steel, fg=Meteorite}
\setbeamercolor{alerted text}{fg=Ruby}

\setbeamercolor{block title}{bg=Meteorite, fg=White}
\setbeamercolor{block title example}{bg=Ocean, fg=White}
\setbeamercolor{block title alerted}{bg=Citrus, fg=White}
\setbeamercolor{block body}{bg=CanvasBG}

\metroset{titleformat=smallcaps, progressbar=foot}

\makeatletter
\setlength{\metropolis@progressinheadfoot@linewidth}{2pt}
\setlength{\metropolis@titleseparator@linewidth}{2pt}
\setlength{\metropolis@progressonsectionpage@linewidth}{2pt}
%%%%%%%%%%%%%%%%%%%%%%%%%%%%
%% Theme Adjustments %%
%%%%%%%%%%%%%%%%%%%%%%%%%%%%


\title{Online Anomaly Localization in Images}
\subtitle{Project Progress Update}
\date{July 19, 2021}
\author{Colin Burdine}
\institute{SULI Intern, Hosted at Argonne National Laboratory}
% \titlegraphic{\hfill\includegraphics[height=1.5cm]{logo.pdf}}

\begin{document}

\maketitle

%\begin{frame}{Table of contents}
%  \setbeamertemplate{section in toc}[sections numbered]
%  \tableofcontents%[hideallsubsections]
%\end{frame}

%\section{Problem Statement Update}

%\begin{frame}{About Myself}
%\begin{center}
%\includegraphics[scale=0.25]{figures/me}
%\end{center}
%\begin{itemize}
%\item I am a recent graduate of Baylor University (Sic 'Em Bears!)
%\item I majored in Mathematics and Computer Science.
%\item I am currently working on my Master's at Baylor, doing research in the field of quantum computing.
%\end{itemize}
%\end{frame}

%\begin{frame}{Problem Statement}
%
%\begin{itemize}
%\item Over the past three to four decades, advances in computer vision systems and machine learning have enabled high-fidelity object detection, classification, and tracking.
%
%\item However, these computer vision models and methods are often quite complex and perform poorly on small computing devices.
%
%\pause 
%\item This poses a challenge if we want to integrate these methods into an \alert{edge computing} system like Waggle:\\[4mm]
%\begin{center}
%\includegraphics[scale=0.4]{figures/waggle_devs}
%\end{center}
%
%\end{itemize}
%\end{frame}

%\begin{frame}{Problem Statement}
%
%\begin{itemize}
%\item Last time, I presented my work on a real-time motion detection Waggle plugin.
%
%\item Since then, I have shifted my attention toward the task of online anomaly detection.\\[4mm]
%
%\begin{alertblock}{Requirements and Constraints (Anomaly Detection)}
%\begin{enumerate}
%\item Must be able to detect large anomalies in a still image scene, with a low false positive rate.
%
%\item Must be able to detect multiple anomalies at once and identify their bounding box in the scene.
%
%\item Must be able to adapt to changes in the scene without human intervention and re-calibration.
%\end{enumerate}
%\end{alertblock}
%
%\end{itemize}
%\end{frame}

%\section{Proposed Methods}

\begin{frame}{Problem Statement}
\begin{itemize}
\item \alert{Image anomaly localization} is the task of locating uncommon objects in an image, if they exist.

\item Many existing anomaly detection models are unsupervised and offline, requiring large labeled datasets and human calibration to perform well \cite{wta_detection, attention_anomalies}.

\item In this research, we propose an ``entirely self-taught" model that performs \alert{unsupervised online anomaly localization}.\\[4mm]


\begin{exampleblock}{Dataset Used: NEON Phenocam (RMNP)}
\begin{tabular}{cc}
\tabimg{figures/RMNP}{0.075} & \tabimg{figures/RMNP_timeline}{0.26}
\end{tabular}
\end{exampleblock}

\end{itemize}
\end{frame}

\begin{frame}{Proposed Methods}
\begin{itemize}

\item My operating hypothesis is:\\ 
\alert{\textit{Anomalies can be identified in images with the reconstruction error of a lossy image compression algorithm.}}\\[2cm]
\end{itemize}


\begin{exampleblock}{Example (JPEG-style compression)}
\includegraphics[scale=0.3]{figures/jpeg_image}\includegraphics[scale=0.3]{figures/jpeg_compressed}\includegraphics[scale=0.3]{figures/jpeg_lossmap}
\end{exampleblock}


\end{frame}

%\begin{frame}{Anomaly Detection (JPEG-style compression)}
%
%\begin{itemize}
%
%\item The first method I tried is a variant of JPEG compression.
%
%\item This method performs PCA on the discrete cosine transform of small ``chunks" of an image.
%
%\item Varying the sizes of these ``chunks" can yield different results.\\[1cm]
%
%\end{itemize}
%\pause
%
%\begin{exampleblock}{Example (JPEG-style compression with vertical strips)}
%\includegraphics[scale=0.3]{figures/jpeg_image}\includegraphics[scale=0.3]{figures/jpeg_compressed}\includegraphics[scale=0.3]{figures/jpeg_lossmap}
%\end{exampleblock}
%
%\end{frame}

\begin{frame}{Anomaly Detection (CAEs)}

\begin{itemize}
\item Convolutional Autoencoders (CAEs) are a machine learning model that \textit{learns} a low-dimensional representation of an image by compressing and then decompressing an image \cite{wta_detection, attention_anomalies}.\\[1cm]
\end{itemize}

\begin{exampleblock}{CAE Architecture}
\begin{center}
\includegraphics[scale=0.35]{figures/cae_diagram}
\end{center}
\end{exampleblock}

\end{frame}

%\begin{frame}{Dataset (NEON Phenocam Dataset)}
%
%\begin{itemize}
%\item Although we would like to deploy this model on Waggle nodes, we did not have a large dataset of stillframe data.\\[2mm]
%
%\item To train an autoencoder model, I used NEON's extensive phenocam dataset:\\
%(\texttt{https://www.neonscience.org/data-collection/phenocams})\\[2mm]
%
%\item As a validation test, I used photo editing software to manually insert anomalies of various sizes into the scene.\\[1cm]
%\end{itemize}
%
%\begin{exampleblock}{Neon Phenocam Dataset Example (RMNP)}
%\begin{tabular}{cc}
%\tabimg{figures/RMNP}{0.06} & \tabimg{figures/RMNP_timeline}{0.26}
%\end{tabular}
%\end{exampleblock}
%
%\end{frame}

%\begin{frame}{Anomaly Detection and Localization}
%\begin{itemize}
%\item In order to localize anomalies, I fitted the reconstruction loss (computed via MSE) to a per-pixel gamma distribution to account for background noise.
%
%\item Anomalies can then be localized using a ``per-pixel p-value test", with an estimated anomaly probability $\alpha$.
%
%\end{itemize}
%
%\begin{exampleblock}{Gamma Filter Example (NEON RMNP Dataset)}
%\includegraphics[scale=0.4]{figures/gamma_filter}
%\includegraphics[scale=0.4]{figures/gamma_filter_noise}
%\end{exampleblock}
%\end{frame}

\begin{frame}{Applying Attention Expansion}

%\begin{itemize}
%\item To increase the true positive rate, we can induce an attention bias to the model by employing \textit{attention expansion loss}.\\[0.5cm]

\begin{exampleblock}{Example: Attention Heatmaps \cite{grad_cam}}
\begin{tabular}{c | c}
\textbf{No Attention Expansion} & \textbf{Attention Expansion}\\
\includegraphics[scale=0.155]{figures/acae_heatmaps_0} & \includegraphics[scale=0.14]{figures/acae_heatmaps}
\end{tabular}
\end{exampleblock}
%\end{itemize}
\end{frame}

\begin{frame}{Results}
\begin{exampleblock}{Examples: Detected Anomalies}
\includegraphics[scale=0.16]{figures/anomalies}
\end{exampleblock}

\begin{exampleblock}{Online Anomaly Detection ROC Curve}
\begin{tabular}{c | c}
\textbf{No Attention Expansion} & \textbf{Attention Expansion}\\
\includegraphics[scale=0.35]{figures/neon_cae_roc} & \includegraphics[scale=0.35]{figures/neon_acae_roc} \\
Area under Curve:  0.6371 & Area under curve: 0.738 \\ 
\end{tabular}
\end{exampleblock}
\end{frame}


%\section{Future Work}
%
%
%\begin{frame}{Future Work}
%
%
%My current priorities for the anomaly detection model are:
%
%\begin{enumerate}
%\item Integrate the plugin prototype with PyWaggle so that it can publish data to other plugins.
%
%\item Implement an online simulator that feeds the model camera data sequentially
%
%\item Implement mechanisms to allow for online training and calibration.
%\end{enumerate}
%
%\end{frame}

{\setbeamercolor{palette primary}{fg=White, bg=Meteorite}
\begin{frame}[standout]
  Questions?\\[2mm]
  Reach out to me at:\ \texttt{cburdine@anl.gov}\\[1cm]
  \small{My Daily Progress Log:
  \href{https://github.com/waggle-sensor/summer2021/tree/main/Burdine}{https://github.com/waggle-sensor/summer2021/tree/main/Burdine}\\[6mm]
  Anomaly Detection Notebook Repository:\\[2mm]
  \href{https://github.com/waggle-sensor/anomaly-detection}{https://github.com/waggle-sensor/anomaly-detection}}
\end{frame}}


\begin{frame}[allowframebreaks]
\frametitle{References}
\bibliographystyle{amsalpha}
\bibliography{project_mid}
\end{frame}
%
%\appendix
%
%\section{Appendix}
%
%\begin{frame}{YOLOv1 Architecture}
%\begin{center}
%\includegraphics[scale=0.3]{figures/yolo_cnn}\\
%(This image is from the original YOLO paper \cite{YOLO})
%\end{center}
%\end{frame}


\end{document}
